 % -> commentaar

% De documentclass 'article' is de beste voor bijna elke situatie.
\documentclass{article}

% Packages:
\usepackage[a4paper]{geometry} % Eenvoudig aanpassen van paginamaat en marges.
\usepackage[dutch]{babel} % Juiste afbreekregels en dergelijke!
\usepackage{parskip} % Alinea's beginnen links uitgelijnd en er staat een lege regel tussen alinea's.
\usepackage{amsmath, amssymb, textcomp} % Wiskundige symbolen e.d.
\usepackage{color} % Kleuren
\usepackage{graphicx} % Plaatjes
\usepackage{enumerate} % Voor opsommingen
\usepackage{hyperref}
\usepackage[official]{eurosym}


\newcommand{\beqn}{\begin{align*}}
\newcommand{\eeqn}{\end{align*}}

% Hier begint het echte document.
\begin{document}

\title{\LaTeX workshop (Exercises)}
\author{De Leidsche Flesch}
\date{\today}

\maketitle

\section{New document}
	\begin{enumerate}
		\item Start a \LaTeX document with the text: \emph{Hello world!}
	\end{enumerate}
	
\section{Text}
	\begin{enumerate}
		\item On the internet, find a short explanation of the definition of \LaTeX\ (a sentence), and cite this sentence. Use a \verb+\footnote{}+ to make a reference to your source, and if necessary a \verb+\url{}+ (don't forget to include the `url' package).
		\item Look up (on the internet) how to make accents, such as co\"ordinate, caf\'e (Dutch word), cura\c{c}ao\"enaar (Dutch word for inhabitant of Cura\c{c}ao), etc. Also, look at the difference between 'text' and `text' (pay attention to the accents).
		\item Some characters, such as \{ are already defined within \LaTeX. How do you think you can portray these characters in a text in the PDF? Hint: take a look at the standard form of the commands.
		\item Try to produce the different list structures (enumerations, etc.) and all things explained in the manual yourself and try to, for example, make a new custom list structure.
	\end{enumerate}


\section{Mathematical environment}
\begin{enumerate}
	\item Reproduce the following formulas, mind the braces!
	\begin{itemize}
		\renewcommand{\labelitemi}{\(\circ\)}
			\item\(a_{1,1} + a_{1,2} + \ldots + a_{1,n} = \displaystyle\sum_{i = 1}^n a_{1,i}\)\\
			\item\( 1 \in \left\{ x~|~\mathbb{R} \backslash 2^{3^4} \right\}\)\\
			\item\(\displaystyle\lim_{n \to \infty} 2^{-n} = 0\)\\
			\item\(\log_2(x\cdot y) \lor \log_4(x\cdot y)\)
		\end{itemize}
		
\end{enumerate}
\section{Do your own research}
\begin{enumerate}
	\item Try the different examples explained in the manual yourself.
	\item Reproduce the following formulas:
	\begin{enumerate}
		\item \[  \left\{(a,b) \in \mathbb{Z}^2 : b \neq 0\right\} \ni (0,1) \]
		\item \[\overrightarrow{AB_{\pm}} = \langle a, \pm b \rangle \neq a\mathbf{i} \mp b\mathbf{j}\]
		\item \euro 42,- (you need a package for this)
		\item Pay attention to the space between \textquotesingle\(\exists\)\textquotesingle \ and \textquotesingle\(\eta\)\textquotesingle! Hint: use \verb|\stackrel{...}{...}| and \verb.\mathcal.
			\[\exists \ \eta :\mathcal{A} \hookrightarrow \mathcal{B}, \zeta : \mathcal{B} \hookrightarrow \mathcal{A}\]
			\[\Updownarrow\]
			\[\exists \ \beta : \mathcal{A} \stackrel{\sim}{\longrightarrow} \mathcal{B}\]
		\item \verb+\underbrace+
			\[\forall \ A, B \in V :  \underbrace{\neg(A \wedge B)}_{\text{not } A \text{ and } B} \Longleftrightarrow \underbrace{(\neg A) \vee (\neg B)}_{\text{not } A \text{ nor } B}\]
		\item 			\[f : A \cup B \to \{0,1\} \text{ with } A \cap B = \emptyset \text{ defined by } x \mapsto \begin{cases} 0 & \text{ if } x \in A\\1 & \text{ if } x \in B \end{cases}\]
		\item 
			\[\binom{k}{n} = \prod_{l = 1}^n\frac{k-l+1}{l}\]
		\item	
			\[\Omega \setminus \left[\bigcup_{i \in I}\left(\bigcup_{j \in J} A_{i,j} \right)\right] \subseteq \left(\bigcap_{\begin{smallmatrix} i \in I \\ j \in J \end{smallmatrix}} A_{i,j}\right)^c\]
		\item
			\[A = \left.\left(
			\begin{array}{cccccc}
				\dot{t} 	& 0 		& 0		& \ldots 	& 0 		& 0\\
				0 		& t 		& 0		& \ldots 	& 0 		& 0\\
				0 		& 0 		& \dot{t} & \ldots 	& 0 		& 0\\
				\vdots 	& \vdots 	& \vdots 	& \ddots 	& \vdots 	& \vdots \\
				0 		& 0 		& 0 		& \ldots 	& t 		& 0\\
				0 		& 0 		& 0 		& \ldots 	& 0 		& \dot{t}
			\end{array}\right)\right|_{t = 0}\]
	\end{enumerate}
\end{enumerate}
\end{document} % Hier eindigt het document.
